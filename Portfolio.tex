\documentclass{article}

\title{Portfolio of MSSP Projects}
\author{Zichun Liu}

\begin{document}
	\maketitle
	\section{Overview}
	\paragraph{} As an important graduation requirement of MSSP, the portfolio is made up of project analysis reports that have been done within the one and a half year of the master program. In this portfolio, I archived eight main projects:
	\begin{itemize}
		\item Ant anatomy and behavior research
		\item Catalog study on women emancipation
		\item Salmon habitation preference study
		\item Measuring the impact of geriatric care among Medicare beneficiaries
		\item R package: Single cell toolkit
		\item Random Contamination and Select Response Styles affecting measures of fit and reliability in factor analysis
		\item R package: celda - Cellular Latent Dirichlet Allocation(ongoing)
		\item Recommender system on Consumer Unit dataset(ongoing)		
	\end{itemize}

	\section{Ant anatomy and behavior research}
	\subsection{Introduction \& Background}
	some context for the project\\
	summarize the basic problem or question that you were asked to address by your client\\
	the nature of the data\\ 
	the goal(s) of your work.
	\subsection{Data Format}
	What is the nature of the data that you used in pursuing your goals\\
	How were they obtained\\
	What, if any, processing, cleaning, etc. was done\\
	Importantly, to what extent – and why – do you expect that these data have information in them relevant to your stated goal(s)?\\
	\subsection{Modeling \& Analysis}
	Describe (concisely!) the key steps you took and the findings you obtained in the modeling and analysis of your data.\\
	\\
	These might include, for example, exploratory data analysis (e.g., visualizations, summary statistics, and initial attempts at modeling),\\
	\\
	and further key steps in the potentially iterative process you pursued. \\
	\\
	For each key step and/or result, there should correspond relevant R code, archived in an appropriate project folder on github, with the project folder referenced appropriately in your portfolio1.
		\subsubsection{Exploratory Data Analysis}
	\subsection{Findings}
	\subsection{Discussion}
	\subsection{Appendix}
	dfalsdjf\\
	adfsdafasdf\\
	asdfsadf\\
	
	
	
	
	\section{Measuring the impact of geriatric care among Medicare beneficiaries}
	\subsection{Introduction \& Background} 
	
	This is a collaboration project with the Trinity Partners. 
	
	Rising medical cost due to aging population is one of the top concerns in the US.  There are currently almost 50 million people over 65 and 70 percent of them have two or more chronic conditions, which contribute to non-stopping, continuing costs of medicine and other medical services. This population cost Medicare nearly 650 billion dollars just looking at year 2015.  
	
	As the result, doctors who focus on elderly patients, geriatricians are gaining interest, both for the system and general public, to see whether geriatric care can provide better outcomes, either clinical or economical, for treating the elderly population. 
	
	Thus, our research focus on three main questions: 
	\begin{itemize}
		\item Would geriatric care increase the longevity of the elderly population? 
		\item Would geriatric care reduce the costs for the Medicare system?
		\item Would geriatric care reduce the medical resources utilization?
	\end{itemize}
	   	
	\subsection{Data Format}
	\paragraph{}The data we have was originated from Medicare claim data from Jan 1st 2010 to Dec 31st 2014, in which each observation is a claim report in the system. \\
	
    For the purpose of studying the impact of geriatric care on patients, we converted the claim-based data to patient-based data. The number of claims in 2010 and CCI are calculated from the information between 2010-01-01 and 2010-12-31. Other variables are calculated or extracted from information between 2011-01-01 and 2014-12-31. \\
    
    2 pair of sample datasets have been used in the analysis, being one for survival analysis and the other for cost and resource analysis.\\

	
	\subsection{Modeling \& Analysis}
	Describe (concisely!) the key steps you took and the findings you obtained in the modeling and analysis of your data.\\
	\\
	These might include, for example, exploratory data analysis (e.g., visualizations, summary statistics, and initial attempts at modeling),\\
	\\
	and further key steps in the potentially iterative process you pursued. \\
	\\
	For each key step and/or result, there should correspond relevant R code, archived in an appropriate project folder on github, with the project folder referenced appropriately in your portfolio1.
	\subsubsection{Exploratory Data Analysis}
	\subsection{Findings}
	\subsection{Discussion}
	\subsection{Appendix}
	
	
	
	\section[section3]{Ant anatomy and behavior research}
	
\end{document}